\documentclass[author-year, review, 11pt]{components/elsarticle} %review=doublespace preprint=single 5p=2 column
%%% Begin My package additions %%%%%%%%%%%%%%%%%%%
\usepackage[hyphens]{url}
\usepackage{lineno} % add
  \linenumbers % turns line numbering on
\bibliographystyle{elsarticle-harv}
\biboptions{sort&compress} % For natbib
\usepackage{graphicx}
\usepackage{booktabs} % book-quality tables
%% Redefines the elsarticle footer
\makeatletter
\def\ps@pprintTitle{%
 \let\@oddhead\@empty
 \let\@evenhead\@empty
 \def\@oddfoot{\it \hfill\today}%
 \let\@evenfoot\@oddfoot}
 \def\tightlist{}
\makeatother

% A modified page layout
\textwidth 6.75in
\oddsidemargin -0.15in
\evensidemargin -0.15in
\textheight 9in
\topmargin -0.5in
%%%%%%%%%%%%%%%% end my additions to header

\usepackage[T1]{fontenc}
\usepackage{lmodern}
\usepackage{amssymb,amsmath}
\usepackage{ifxetex,ifluatex}
\usepackage{fixltx2e} % provides \textsubscript
% use upquote if available, for straight quotes in verbatim environments
\IfFileExists{upquote.sty}{\usepackage{upquote}}{}
\ifnum 0\ifxetex 1\fi\ifluatex 1\fi=0 % if pdftex
  \usepackage[utf8]{inputenc}
\else % if luatex or xelatex
  \usepackage{fontspec}
  \ifxetex
    \usepackage{xltxtra,xunicode}
  \fi
  \defaultfontfeatures{Mapping=tex-text,Scale=MatchLowercase}
  \newcommand{\euro}{€}
\fi
% use microtype if available
\IfFileExists{microtype.sty}{\usepackage{microtype}}{}
\usepackage{color}
\usepackage{fancyvrb}
\newcommand{\VerbBar}{|}
\newcommand{\VERB}{\Verb[commandchars=\\\{\}]}
\DefineVerbatimEnvironment{Highlighting}{Verbatim}{commandchars=\\\{\}}
% Add ',fontsize=\small' for more characters per line
\usepackage{framed}
\definecolor{shadecolor}{RGB}{248,248,248}
\newenvironment{Shaded}{\begin{snugshade}}{\end{snugshade}}
\newcommand{\AlertTok}[1]{\textcolor[rgb]{0.94,0.16,0.16}{#1}}
\newcommand{\AnnotationTok}[1]{\textcolor[rgb]{0.56,0.35,0.01}{\textbf{\textit{#1}}}}
\newcommand{\AttributeTok}[1]{\textcolor[rgb]{0.77,0.63,0.00}{#1}}
\newcommand{\BaseNTok}[1]{\textcolor[rgb]{0.00,0.00,0.81}{#1}}
\newcommand{\BuiltInTok}[1]{#1}
\newcommand{\CharTok}[1]{\textcolor[rgb]{0.31,0.60,0.02}{#1}}
\newcommand{\CommentTok}[1]{\textcolor[rgb]{0.56,0.35,0.01}{\textit{#1}}}
\newcommand{\CommentVarTok}[1]{\textcolor[rgb]{0.56,0.35,0.01}{\textbf{\textit{#1}}}}
\newcommand{\ConstantTok}[1]{\textcolor[rgb]{0.00,0.00,0.00}{#1}}
\newcommand{\ControlFlowTok}[1]{\textcolor[rgb]{0.13,0.29,0.53}{\textbf{#1}}}
\newcommand{\DataTypeTok}[1]{\textcolor[rgb]{0.13,0.29,0.53}{#1}}
\newcommand{\DecValTok}[1]{\textcolor[rgb]{0.00,0.00,0.81}{#1}}
\newcommand{\DocumentationTok}[1]{\textcolor[rgb]{0.56,0.35,0.01}{\textbf{\textit{#1}}}}
\newcommand{\ErrorTok}[1]{\textcolor[rgb]{0.64,0.00,0.00}{\textbf{#1}}}
\newcommand{\ExtensionTok}[1]{#1}
\newcommand{\FloatTok}[1]{\textcolor[rgb]{0.00,0.00,0.81}{#1}}
\newcommand{\FunctionTok}[1]{\textcolor[rgb]{0.00,0.00,0.00}{#1}}
\newcommand{\ImportTok}[1]{#1}
\newcommand{\InformationTok}[1]{\textcolor[rgb]{0.56,0.35,0.01}{\textbf{\textit{#1}}}}
\newcommand{\KeywordTok}[1]{\textcolor[rgb]{0.13,0.29,0.53}{\textbf{#1}}}
\newcommand{\NormalTok}[1]{#1}
\newcommand{\OperatorTok}[1]{\textcolor[rgb]{0.81,0.36,0.00}{\textbf{#1}}}
\newcommand{\OtherTok}[1]{\textcolor[rgb]{0.56,0.35,0.01}{#1}}
\newcommand{\PreprocessorTok}[1]{\textcolor[rgb]{0.56,0.35,0.01}{\textit{#1}}}
\newcommand{\RegionMarkerTok}[1]{#1}
\newcommand{\SpecialCharTok}[1]{\textcolor[rgb]{0.00,0.00,0.00}{#1}}
\newcommand{\SpecialStringTok}[1]{\textcolor[rgb]{0.31,0.60,0.02}{#1}}
\newcommand{\StringTok}[1]{\textcolor[rgb]{0.31,0.60,0.02}{#1}}
\newcommand{\VariableTok}[1]{\textcolor[rgb]{0.00,0.00,0.00}{#1}}
\newcommand{\VerbatimStringTok}[1]{\textcolor[rgb]{0.31,0.60,0.02}{#1}}
\newcommand{\WarningTok}[1]{\textcolor[rgb]{0.56,0.35,0.01}{\textbf{\textit{#1}}}}
\usepackage{longtable}
\ifxetex
  \usepackage[setpagesize=false, % page size defined by xetex
              unicode=false, % unicode breaks when used with xetex
              xetex]{hyperref}
\else
  \usepackage[unicode=true]{hyperref}
\fi
\hypersetup{breaklinks=true,
            bookmarks=true,
            pdfauthor={},
            pdftitle={rOpenSci tools for accessing science literature for textmining},
            colorlinks=true,
            urlcolor=blue,
            linkcolor=magenta,
            pdfborder={0 0 0}}
\urlstyle{same}  % don't use monospace font for urls
\setlength{\parindent}{0pt}
\setlength{\parskip}{6pt plus 2pt minus 1pt}
\setlength{\emergencystretch}{3em}  % prevent overfull lines
\setcounter{secnumdepth}{0}
% Pandoc toggle for numbering sections (defaults to be off)
\setcounter{secnumdepth}{0}
% Pandoc header



\begin{document}
\begin{frontmatter}

  \title{rOpenSci tools for accessing science literature for textmining}
    \author[cstar]{Scott Chamberlain\corref{c1}}
   \ead{myrmecocystus(at)gmail.com} 
   \cortext[c1]{Corresponding author}
      \address[cstar]{rOpenSci, Museum of Paleontology, University of California, Berkeley,
CA, USA}
  
  \begin{abstract}
  Corresponding Author:
  
  Scott Chamberlain
  
  rOpenSci, Museum of Paleontology, University of California, Berkeley,
  CA, USA
  
  Email address:
  \href{mailto:myrmecocystus@gmail.com}{\nolinkurl{myrmecocystus@gmail.com}}
  
  \newpage
  
  Background. xxxx.
  
  Methods. xxxx.
  
  Results. xxxx.
  
  Discussion. xxxx.
  \end{abstract}
  
 \end{frontmatter}


\newpage

\hypertarget{introduction}{%
\section{Introduction}\label{introduction}}

There's more than 100 million articles published (source: Crossref API),
representing an enormous amount of knowledge. In addition to simply
reading these articles, they contain a vast trove of information of
interest to researchers for machine aided questions.

For example, many researchers are interested in statistical outcomes of
articles: questions about P-values, about effect sizes, and more. With
regard to effect sizes, these are of particular interest, as they are
often combined in meta-analyses to draw broad conclusions about a
particular question.

Text-mining is the broad term associated with pulling information out of
articles. Given the importance of text-mining, good text-mining tools
are needed to make it easier for researchers. In particular, the R
programming language is used widely throughout many academic fields and
thus tools in R for text mining are of particular importance.

Here, we present an overview of text-mining tools in the R programming
language. We do not cover analysis tools per se, but rather those tools
for searching for, acquiring, and ``mashing up'' text.

\hypertarget{digital-articles-technical-aspects}{%
\section{Digital articles: technical
aspects}\label{digital-articles-technical-aspects}}

Those articles that are digital can be split into two groups: easily
machine readable and non-machine readable.

The machine readable articles are those in XML, JSON, or plain text
format. The former two, XML and JSON, are ideal for the machine readable
types because they are structured data, whereas plain text has no
structure - it's simply a set of characters with line breaks and spaces
in between.

Of the non-machine readable kind, there's PDFs. These can be broken out
into two groups: text based PDFs and scanned PDFs. The former are
converted from digital versions of various kinds (MS Word, OpenOffice,
markdown, etc.), while the latter are PDFs created by scanning in print
articles for which there is no digital version.

\hypertarget{digital-articles-the-access-landscape}{%
\section{Digital articles: the access
landscape}\label{digital-articles-the-access-landscape}}

Acces to full-text is the holy grail in text-mining. Some use cases can
get by with article metadata (authors, title, etc.), some with
abstracts, but many use cases need full-text.

The landscape of access to full-text is extremely hetergeous, with the
majority of variation along the publisher axis. The major hurdle is
paywalls. The majority of articles are published by the big three
publishers - Wiley, Springer, Elsevier - and the majority of their
articles are behind paywalls.

A promising sign is an increasing number of open access publishers, yet
these represent a very small portion of the total articles (XXXXX)
(ref.).

With respect to paywalled articles, access varies by institution,
depending on what each institution decided to pay for. In addition, some
users may not realize access varies with IP address so that access from
campus vs.~from home (if not on a VPN) will drastically differ.

We can not end this section without mentioning SciHub. This is a last
resort option for many probably, providing access to full text of
articles that are normally paywalled. No tools in this manuscript
provide access to SciHub.

\hypertarget{the-discovery-problem}{%
\section{The discovery problem}\label{the-discovery-problem}}

xxx

\hypertarget{section}{%
\section{}\label{section}}

xxx

\hypertarget{data-sources}{%
\section{Data sources}\label{data-sources}}

There is increasing open access scientific literature content available
online. However, only a small proportion of scientific journals provide
access to their full content; whereas, most publishers provide open
access to their metadata only (most often through Crossref; Table 1).
The following is a synopsis of the major data sources and associated R
tools.

\newpage

Table 1. Sources of scientific literature, their content type provided
via web services, whether rOpenSci has an R packages for the service,
and where to find the API documentation.

\begin{longtable}[]{@{}llll@{}}
\toprule
Data Provider & Content Type & rOpenSci Package &
Documentation\tabularnewline
\midrule
\endhead
Crossref & Metadata only & rcrossref/crminer & \footnote{\url{https://api.crossref.org}}\tabularnewline
DataCite & Metadata only & rdatacite & \footnote{\url{https://support.datacite.org/docs/api}}\tabularnewline
Biodiversity Heritage Library & Full content/Metadata & rbhl &
\footnote{\url{http://bit.ly/KYQ1Rd}}\tabularnewline
Public Library of Science (PLoS) & Full text/altmetrics & rplos &
\footnote{\url{http://api.plos.org/solr}}\tabularnewline
Scopus (Elsevier) & Full content/Metadata & fulltext & \footnote{\url{http://bit.ly/J9S616}}\tabularnewline
arXiv & Full content/Metadata & aRxiv & \footnote{\url{https://arxiv.org/help/api/index}}\tabularnewline
Biomed Central (via Springer) & Full content/Metadata & fulltext &
\footnote{\url{https://dev.springer.com/}}\tabularnewline
bioRxiv & Full content/Metadata & fulltext & \footnote{\url{http://www.biorxiv.org/}}\tabularnewline
PMC/Pubmed (via Entrez) & Full content/Metadata & rentrez & \footnote{\url{https://www.ncbi.nlm.nih.gov/books/NBK25500}}\tabularnewline
Europe PMC & Full content/Metadata & europepmc & \footnote{\url{https://azure.microsoft.com/en-us/services/cognitive-services}}\tabularnewline
Microsoft Academic Search & Metadata & fulltext/microdemic & \footnote{\url{https://dev.labs.cognitive.microsoft.com/docs/services/56332331778daf02acc0a50b/operations/565d9001ca73072048922d97}}\tabularnewline
Directory of Open Access Journals & Metadata & jaod & \footnote{\url{https://doaj.org/api/v1/docs}}\tabularnewline
JSTOR Data for Research & Full content & jstor & \footnote{\url{https://www.jstor.org/dfr/}}\tabularnewline
ORCID & Metadata & rorcid & \footnote{\url{https://pub.orcid.org/}}\tabularnewline
Wikimedia's Citoid & Citations & rcitoid & \footnote{\url{https://en.wikipedia.org/api/rest_v1/\#/Citation/getCitation}}\tabularnewline
Open Citation Corpus & Citations & citecorp & \footnote{\url{http://opencitations.net/}}\tabularnewline
\bottomrule
\end{longtable}

\hypertarget{crossrefdatacite}{%
\subsection{Crossref/Datacite}\label{crossrefdatacite}}

Crossref is a non-profit that creates (or ``mints'') Digital Object
Identifiers (DOIs). In addition, they maintain metadata associated with
each DOI. The metadata ranges from simple (including author, title,
dates, DOI, type, publisher) to including number of citations to the
article, as well as references in the article, and even abstracts. At
the time of writing they hold 100 million DOIs.

One can search by DOI or search citation data to get citations. In
addition, Crossref has a text-mining opt-in program for publishers. The
result of this is that some publishers provide URLs for full text
content of their articles. The majority of these links are pay-walled,
while some are open access. Using any of the various tools for working
with Crossref data, you can filter your search to get only articles with
full text links, and further to get only articles with full text links
that are open access.

The main interfaces for Crossref in R are
\href{https://github.com/ropensci/rcrossref}{rcrossref} and
\href{https://github.com/ropensci/crminer}{crminer}. Similar interfaces
are available in Ruby
(\href{https://github.com/sckott/serrano}{serrano}) and Python
(\href{https://github.com/sckott/habanero}{habanero}).

Datacite is similar to Crossref, but focuses on datasets instead of
articles. The main interface for Datacite in R is
\href{https://github.com/ropensci/rdatacite}{rdatacite}.

\hypertarget{biodiversity-heritage-library}{%
\subsection{Biodiversity Heritage
Library}\label{biodiversity-heritage-library}}

The Biodiversity Heritage Library (BHL) houses scans of biodiversity
books, and provides web interfaces and APIs to query and fetch those
data. They also provide text of the scanned pages. The main R interace
to BHL is through \href{https://github.com/ropensci/rbhl}{rbhl}.

\hypertarget{public-library-of-science}{%
\subsection{Public Library of Science}\label{public-library-of-science}}

The Public Library of Science (PLOS) is one of the largest open access
only publishers. They as of this writing have published 2.1 million
articles. One of the strongs advantages of PLOS is that they provide an
API to their Solr instance, which is a very flexible way to search their
articles. The main R interace to PLOS is through
\href{https://github.com/ropensci/rplos}{rplos}.

\hypertarget{elsevierscopus}{%
\subsection{Elsevier/Scopus}\label{elsevierscopus}}

Elsevier is one of the largest publishers. Most of their articles are
not open access. However, they have a numbrer of advantages if you have
access to their articles: they are one of the few publishers to provide
machine readable XML (many publishers do have XML versions of articles,
but do not provide it); they are one of the few (two) publishers part of
Crossref's text and data mining program. The packages
\href{https://github.com/ropensci/fulltext}{fulltext} and
\href{https://github.com/ropensci/crminer}{crminer} can be used to
access Elsevier articles through Crossref's TDM program. There's an
interface to Scopus article search within
\href{https://github.com/ropensci/fulltext}{fulltext}.

\hypertarget{arxivbiorxiv}{%
\subsection{arXiv/bioRxiv}\label{arxivbiorxiv}}

arXiv and bioRxiv are preprint publishers, the former in existence for
many years, and the latter new on the scene. You can access articles
from these publishers through
\href{https://github.com/ropensci/fulltext}{fulltext}. arXiv does
provide a web API that we hook into; bioRxiv does not, but we can get
you articles nonetheless.

\hypertarget{pubmedpmceurope-pmc}{%
\subsection{Pubmed/PMC/Europe PMC}\label{pubmedpmceurope-pmc}}

Pubmed/PMC is a corpus/website of NIH funded research in the United
States; while Europe PMC is an equivalent for the European Union. You
can access articles from Pubmed/PMC through
\href{https://github.com/ropensci/fulltext}{fulltext}, and for Europe
PMC through \href{https://github.com/ropensci/europepmc}{europepmc}.

\hypertarget{microsoft-academic-research}{%
\subsection{Microsoft Academic
Research}\label{microsoft-academic-research}}

Microsoft Academic Research (MAR) is a search engine for research
articles. You can use their GUI web interface to search, and they
provide APIs for programmatic access. The R interface for MAR is
\href{https://github.com/ropensci/microdemic}{microdemic}; and
\href{https://github.com/ropensci/fulltext}{fulltext} hooks into
\texttt{microdemic} as well for article search and abstract retrieval.

\hypertarget{directory-of-open-access-journals}{%
\subsection{Directory of Open Access
Journals}\label{directory-of-open-access-journals}}

xxxxx

\hypertarget{jstor}{%
\subsection{JSTOR}\label{jstor}}

xxxxx

\hypertarget{orcid}{%
\subsection{ORCID}\label{orcid}}

xxxxx

\hypertarget{citoidopen-citation-corpus}{%
\subsection{Citoid/Open Citation
Corpus}\label{citoidopen-citation-corpus}}

xxx

\hypertarget{how-to-text-mine-from-r-three-case-studies}{%
\section{How to text mine from R: Three case
studies}\label{how-to-text-mine-from-r-three-case-studies}}

\hypertarget{case-study-1-citation-mining}{%
\subsubsection{Case study 1: Citation
mining}\label{case-study-1-citation-mining}}

In this example, xxxx

\emph{Load libraries}

\begin{Shaded}
\begin{Highlighting}[]
\KeywordTok{library}\NormalTok{(}\StringTok{"rcrossref"}\NormalTok{)}
\KeywordTok{library}\NormalTok{(}\StringTok{"rplos"}\NormalTok{)}
\KeywordTok{library}\NormalTok{(}\StringTok{"rorcid"}\NormalTok{)}
\KeywordTok{library}\NormalTok{(}\StringTok{"rcitoid"}\NormalTok{)}
\KeywordTok{library}\NormalTok{(}\StringTok{"citecorp"}\NormalTok{)}
\end{Highlighting}
\end{Shaded}

\emph{rcrossref}

Using \texttt{rcrossref} for Crossref data:

\begin{Shaded}
\begin{Highlighting}[]
\NormalTok{x <-}\StringTok{ }\KeywordTok{cr_works}\NormalTok{(}\DataTypeTok{query=}\StringTok{"NSF"}\NormalTok{)}
\KeywordTok{head}\NormalTok{(x}\OperatorTok{$}\NormalTok{data)}
\CommentTok{#> # A tibble: 6 x 32}
\CommentTok{#>   alternative.id container.title created deposited published.print doi  }
\CommentTok{#>   <chr>          <chr>           <chr>   <chr>     <chr>           <chr>}
\CommentTok{#> 1 S106352031630~ Applied and Co~ 2016-0~ 2019-02-~ 2018-03         10.1~}
\CommentTok{#> 2 <NA>           Biogeosciences~ 2017-0~ 2017-07-~ <NA>            10.5~}
\CommentTok{#> 3 <NA>           Global Biogeoc~ 2018-0~ 2019-01-~ 2018-10         10.1~}
\CommentTok{#> 4 <NA>           IEEE Communica~ 2016-1~ 2017-12-~ 2017            10.1~}
\CommentTok{#> 5 S002178241400~ Journal de Mat~ 2014-0~ 2018-10-~ 2014-10         10.1~}
\CommentTok{#> 6 123            Light: Science~ 2019-0~ 2019-01-~ 2019-12         10.1~}
\CommentTok{#> # ... with 26 more variables: indexed <chr>, issn <chr>, issue <chr>,}
\CommentTok{#> #   issued <chr>, member <chr>, page <chr>, prefix <chr>, publisher <chr>,}
\CommentTok{#> #   reference.count <chr>, score <chr>, ...}
\end{Highlighting}
\end{Shaded}

\hypertarget{case-study-2-abstract-mining}{%
\subsubsection{Case study 2: Abstract
mining}\label{case-study-2-abstract-mining}}

Sometimes you just need abstracts for your research question. The
benefit of only needing abstracts, and not need full text, is that
there's many more articles that will have abstracts available than have
their full text available.

As an example, let's say you xxxx

\begin{Shaded}
\begin{Highlighting}[]
\KeywordTok{library}\NormalTok{(}\StringTok{"fulltext"}\NormalTok{)}
\end{Highlighting}
\end{Shaded}

\emph{xxxxx}

Using \texttt{fulltext}:

\begin{Shaded}
\begin{Highlighting}[]
\NormalTok{res <-}\StringTok{ }\KeywordTok{ft_search}\NormalTok{(}\StringTok{"ecology"}\NormalTok{, }\DataTypeTok{from =} \StringTok{"crossref"}\NormalTok{,}
  \DataTypeTok{crossrefopts =} \KeywordTok{list}\NormalTok{(}\DataTypeTok{filter =} \KeywordTok{c}\NormalTok{(}\DataTypeTok{has_abstract =} \OtherTok{TRUE}\NormalTok{)))}
\NormalTok{ids <-}\StringTok{ }\NormalTok{res}\OperatorTok{$}\NormalTok{crossref}\OperatorTok{$}\NormalTok{data}\OperatorTok{$}\NormalTok{doi}
\NormalTok{out <-}\StringTok{ }\KeywordTok{ft_abstract}\NormalTok{(}\DataTypeTok{x =}\NormalTok{ ids, }\DataTypeTok{from =} \StringTok{"crossref"}\NormalTok{)}
\NormalTok{abstracts <-}\StringTok{ }\KeywordTok{vapply}\NormalTok{(out}\OperatorTok{$}\NormalTok{crossref, }\StringTok{"[["}\NormalTok{, }\StringTok{""}\NormalTok{, }\StringTok{"abstract"}\NormalTok{)}
\end{Highlighting}
\end{Shaded}

Using \href{https://quanteda.io/}{quanteda}, read the abstracts into a
corpus

\begin{Shaded}
\begin{Highlighting}[]
\KeywordTok{library}\NormalTok{(}\StringTok{"quanteda"}\NormalTok{)}
\NormalTok{corp <-}\StringTok{ }\KeywordTok{corpus}\NormalTok{(abstracts)}
\KeywordTok{docvars}\NormalTok{(corp) <-}\StringTok{ }\NormalTok{ids}
\end{Highlighting}
\end{Shaded}

Get a summary of the abstracts

\begin{Shaded}
\begin{Highlighting}[]
\KeywordTok{summary}\NormalTok{(corp)}
\CommentTok{#> Corpus consisting of 10 documents:}
\CommentTok{#> }
\CommentTok{#>    Text Types Tokens Sentences                    V1}
\CommentTok{#>   text1   143    262        10   10.2458/v22i1.21112}
\CommentTok{#>   text2   117    244         6   10.2458/v17i1.21696}
\CommentTok{#>   text3    75    118         4   10.2458/v25i1.23119}
\CommentTok{#>   text4     5      8         1    10.2458/v1i1.21154}
\CommentTok{#>   text5   105    171         7   10.1155/2011/868426}
\CommentTok{#>   text6   112    181         6   10.1155/2012/273413}
\CommentTok{#>   text7   117    240         8 10.5194/we-13-91-2013}
\CommentTok{#>   text8   140    245         9 10.5194/we-13-95-2013}
\CommentTok{#>   text9   107    202         7   10.1155/2014/198707}
\CommentTok{#>  text10   118    224         6   10.5402/2011/897578}
\CommentTok{#> }
\CommentTok{#> Source: /Users/sckott/github/ropensci/textmine/use-cases/* on x86_64 by sckott}
\CommentTok{#> Created: Fri Apr  5 11:36:04 2019}
\CommentTok{#> Notes:}
\end{Highlighting}
\end{Shaded}

Use the \texttt{kwic()} function to see a word in context across the
abstracts

\begin{Shaded}
\begin{Highlighting}[]
\KeywordTok{kwic}\NormalTok{(corp, }\DataTypeTok{pattern =} \StringTok{"ecology"}\NormalTok{)}
\CommentTok{#>                                                                         }
\CommentTok{#>   [text1, 33] knowledge production within critical political | ecology |}
\CommentTok{#>   [text1, 50]              in scientific articles on dryland | ecology |}
\CommentTok{#>  [text1, 204]                 to equilibrium models in range | ecology |}
\CommentTok{#>  [text1, 246]    communal areas.Keywords: Critical political | ecology |}
\CommentTok{#>  [text1, 255]                 , scientific models, rangeland | ecology |}
\CommentTok{#>    [text2, 5]                            < jats:p> Political | ecology |}
\CommentTok{#>   [text2, 23]        manifestations of political economy and | ecology |}
\CommentTok{#>   [text2, 45]                      I try to extend political | ecology |}
\CommentTok{#>  [text2, 149]                   , in dialogue with political | ecology |}
\CommentTok{#>  [text2, 177]            people and resources that political | ecology |}
\CommentTok{#>  [text2, 229]      indigeneity scholars.Key words: political | ecology |}
\CommentTok{#>   [text3, 71]                   an analysis from a political | ecology |}
\CommentTok{#>  [text3, 114]                system, supermarkets, political | ecology |}
\CommentTok{#>  [text6, 134]                was observed when allopatry and | ecology |}
\CommentTok{#>  [text7, 167]             ecosystem should be considered for | ecology |}
\CommentTok{#>  [text7, 185]                       the" four-color issue of | ecology |}
\CommentTok{#>  [text7, 201]             step toward advancing knowledge in | ecology |}
\CommentTok{#>  [text9, 195]         or for theoretical studies integrating | ecology |}
\CommentTok{#>                                              }
\CommentTok{#>  . This article is a                         }
\CommentTok{#>  , and investigates the functions            }
\CommentTok{#>  , and the fence-line photographs            }
\CommentTok{#>  , fence-line photography, scientific        }
\CommentTok{#>  , Southern Africa</                         }
\CommentTok{#>  has expanded in multiple new                }
\CommentTok{#>  in the" problem"                            }
\CommentTok{#>  to engage with ethnic studies               }
\CommentTok{#>  approaches to better understand the         }
\CommentTok{#>  focuses on cannot be adequately             }
\CommentTok{#>  , coloniality, Maidu,                       }
\CommentTok{#>  standpoint allows a different interpretation}
\CommentTok{#>  </ jats:p>                                  }
\CommentTok{#>  act together, leading to                    }
\CommentTok{#>  "? Here, I                                  }
\CommentTok{#>  ", and propose that                         }
\CommentTok{#>  and conservation biology. In                }
\CommentTok{#>  and biogeography.</}
\end{Highlighting}
\end{Shaded}

\hypertarget{case-study-3-full-text-mining}{%
\subsubsection{Case study 3: Full text
mining}\label{case-study-3-full-text-mining}}

In this example, xxxx

\begin{Shaded}
\begin{Highlighting}[]
\KeywordTok{library}\NormalTok{(}\StringTok{"fulltext"}\NormalTok{)}
\CommentTok{# library("crminer")}
\end{Highlighting}
\end{Shaded}

\emph{Search for articles}

Search for the term \emph{ecology} in PLOS journals.

\begin{Shaded}
\begin{Highlighting}[]
\NormalTok{(res1 <-}\StringTok{ }\KeywordTok{ft_search}\NormalTok{(}\DataTypeTok{query =} \StringTok{'ecology'}\NormalTok{, }\DataTypeTok{from =} \StringTok{'plos'}\NormalTok{))}
\CommentTok{#> Query:}
\CommentTok{#>   [ecology] }
\CommentTok{#> Found:}
\CommentTok{#>   [PLoS: 47337; BMC: 0; Crossref: 0; Entrez: 0; arxiv: 0; biorxiv: 0; Europe PMC: 0; Scopus: 0; Microsoft: 0] }
\CommentTok{#> Returned:}
\CommentTok{#>   [PLoS: 10; BMC: 0; Crossref: 0; Entrez: 0; arxiv: 0; biorxiv: 0; Europe PMC: 0; Scopus: 0; Microsoft: 0]}
\end{Highlighting}
\end{Shaded}

Each publisher/search-engine has a slot with metadata and data

\begin{Shaded}
\begin{Highlighting}[]
\NormalTok{res1}\OperatorTok{$}\NormalTok{plos}
\CommentTok{#> Query: [ecology] }
\CommentTok{#> Records found, returned: [47337, 10] }
\CommentTok{#> License: [CC-BY] }
\CommentTok{#>                              id}
\CommentTok{#> 1  10.1371/journal.pone.0001248}
\CommentTok{#> 2  10.1371/journal.pone.0059813}
\CommentTok{#> 3  10.1371/journal.pone.0155019}
\CommentTok{#> 4  10.1371/journal.pone.0080763}
\CommentTok{#> 5  10.1371/journal.pone.0208370}
\CommentTok{#> 6  10.1371/journal.pone.0150648}
\CommentTok{#> 7  10.1371/journal.pcbi.1003594}
\CommentTok{#> 8  10.1371/journal.pone.0102437}
\CommentTok{#> 9  10.1371/journal.pone.0175014}
\CommentTok{#> 10 10.1371/journal.pone.0166559}
\end{Highlighting}
\end{Shaded}

\emph{Get full text}

Using the results from \texttt{ft\_search()} we can grab full text of
some articles

\begin{Shaded}
\begin{Highlighting}[]
\NormalTok{(out <-}\StringTok{ }\KeywordTok{ft_get}\NormalTok{(res1))}
\CommentTok{#> <fulltext text>}
\CommentTok{#> [Docs] 10 }
\CommentTok{#> [Source] ext - /Users/sckott/Library/Caches/R/fulltext }
\CommentTok{#> [IDs] 10.1371/journal.pone.0001248 10.1371/journal.pone.0059813}
\CommentTok{#>      10.1371/journal.pone.0155019 10.1371/journal.pone.0080763}
\CommentTok{#>      10.1371/journal.pone.0208370 10.1371/journal.pone.0150648}
\CommentTok{#>      10.1371/journal.pcbi.1003594 10.1371/journal.pone.0102437}
\CommentTok{#>      10.1371/journal.pone.0175014 10.1371/journal.pone.0166559 ...}
\end{Highlighting}
\end{Shaded}

\emph{Extract text from pdfs}

Ideally for text mining you have access to XML or other text based
formats. However, sometimes you only have access to PDFs. In this case
you want to extract text from PDFs. \texttt{fulltext} can help with
that.

You can extract from any pdf from a file path, like:

\begin{Shaded}
\begin{Highlighting}[]
\NormalTok{path <-}\StringTok{ }\KeywordTok{system.file}\NormalTok{(}\StringTok{"examples"}\NormalTok{, }\StringTok{"example1.pdf"}\NormalTok{, }\DataTypeTok{package =} \StringTok{"fulltext"}\NormalTok{)}
\KeywordTok{ft_extract}\NormalTok{(path)}
\CommentTok{#> <document>/Library/Frameworks/R.framework/Versions/3.5/Resources/library/fulltext/examples/example1.pdf}
\CommentTok{#>   Title: Suffering and mental health among older people living in nursing homes---a mixed-methods study}
\CommentTok{#>   Producer: pdfTeX-1.40.10}
\CommentTok{#>   Creation date: 2015-07-17}
\end{Highlighting}
\end{Shaded}

\emph{Extract text chunks}

Requires the \href{https://github.com/ropensci/pubchunks}{pubchunks}
library. Here, we'll search for some PLOS articles, then get their full
text, then extract various parts of each article with
\texttt{pub\_chunks()}.

\begin{Shaded}
\begin{Highlighting}[]
\KeywordTok{library}\NormalTok{(}\StringTok{"pubchunks"}\NormalTok{)}
\NormalTok{res <-}\StringTok{ }\KeywordTok{ft_search}\NormalTok{(}\DataTypeTok{query =} \StringTok{"ecology"}\NormalTok{, }\DataTypeTok{from =} \StringTok{"plos"}\NormalTok{, }\DataTypeTok{limit =} \DecValTok{3}\NormalTok{)}
\NormalTok{x <-}\StringTok{ }\KeywordTok{ft_get}\NormalTok{(res)}
\NormalTok{x }\OperatorTok\StringTok{ }\KeywordTok{ft_collect}\NormalTok{() }\OperatorTok\StringTok{ }\KeywordTok{pub_chunks}\NormalTok{(}\KeywordTok{c}\NormalTok{(}\StringTok{"doi"}\NormalTok{, }\StringTok{"history"}\NormalTok{)) }\OperatorTok\StringTok{ }\KeywordTok{pub_tabularize}\NormalTok{()}
\CommentTok{#> $plos}
\CommentTok{#> $plos$`10.1371/journal.pone.0001248`}
\CommentTok{#>                            doi history.received history.accepted}
\CommentTok{#> 1 10.1371/journal.pone.0001248       2007-07-02       2007-11-06}
\CommentTok{#>   .publisher}
\CommentTok{#> 1       plos}
\CommentTok{#> }
\CommentTok{#> $plos$`10.1371/journal.pone.0059813`}
\CommentTok{#>                            doi history.received history.accepted}
\CommentTok{#> 1 10.1371/journal.pone.0059813       2012-09-16       2013-02-19}
\CommentTok{#>   .publisher}
\CommentTok{#> 1       plos}
\CommentTok{#> }
\CommentTok{#> $plos$`10.1371/journal.pone.0155019`}
\CommentTok{#>                            doi history.received history.accepted}
\CommentTok{#> 1 10.1371/journal.pone.0155019       2015-09-22       2016-04-22}
\CommentTok{#>   .publisher}
\CommentTok{#> 1       plos}
\end{Highlighting}
\end{Shaded}

\hypertarget{future-directions}{%
\section{Future directions}\label{future-directions}}

xxxx

\hypertarget{acknowledgments}{%
\section{Acknowledgments}\label{acknowledgments}}

xxxx

\hypertarget{data-accessibility}{%
\section{Data Accessibility}\label{data-accessibility}}

All scripts and data used in this paper can be found in the permanent
data archive Zenodo under the digital object identifier (DOI). This DOI
corresponds to a snapshot of the GitHub repository at
\url{https://github.com/ropensci/textmine}. Software can be found at
\url{https://github.com/ropensci/xxx}, xxxx, all under MIT licenses.

\hypertarget{references}{%
\section{References}\label{references}}

\end{document}


