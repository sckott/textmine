\documentclass[author-year, review, 11pt]{components/elsarticle} %review=doublespace preprint=single 5p=2 column
%%% Begin My package additions %%%%%%%%%%%%%%%%%%%
\usepackage[hyphens]{url}
\usepackage{lineno} % add
  \linenumbers % turns line numbering on
\bibliographystyle{elsarticle-harv}
\biboptions{sort&compress} % For natbib
\usepackage{graphicx}
\usepackage{booktabs} % book-quality tables
%% Redefines the elsarticle footer
\makeatletter
\def\ps@pprintTitle{%
 \let\@oddhead\@empty
 \let\@evenhead\@empty
 \def\@oddfoot{\it \hfill\today}%
 \let\@evenfoot\@oddfoot}
 \def\tightlist{}
\makeatother

% A modified page layout
\textwidth 6.75in
\oddsidemargin -0.15in
\evensidemargin -0.15in
\textheight 9in
\topmargin -0.5in
%%%%%%%%%%%%%%%% end my additions to header

\usepackage[T1]{fontenc}
\usepackage{lmodern}
\usepackage{amssymb,amsmath}
\usepackage{ifxetex,ifluatex}
\usepackage{fixltx2e} % provides \textsubscript
% use upquote if available, for straight quotes in verbatim environments
\IfFileExists{upquote.sty}{\usepackage{upquote}}{}
\ifnum 0\ifxetex 1\fi\ifluatex 1\fi=0 % if pdftex
  \usepackage[utf8]{inputenc}
\else % if luatex or xelatex
  \usepackage{fontspec}
  \ifxetex
    \usepackage{xltxtra,xunicode}
  \fi
  \defaultfontfeatures{Mapping=tex-text,Scale=MatchLowercase}
  \newcommand{\euro}{€}
\fi
% use microtype if available
\IfFileExists{microtype.sty}{\usepackage{microtype}}{}
\usepackage{longtable}
\ifxetex
  \usepackage[setpagesize=false, % page size defined by xetex
              unicode=false, % unicode breaks when used with xetex
              xetex]{hyperref}
\else
  \usepackage[unicode=true]{hyperref}
\fi
\hypersetup{breaklinks=true,
            bookmarks=true,
            pdfauthor={},
            pdftitle={rOpenSci tools for textmining open source science literature},
            colorlinks=true,
            urlcolor=blue,
            linkcolor=magenta,
            pdfborder={0 0 0}}
\urlstyle{same}  % don't use monospace font for urls
\setlength{\parindent}{0pt}
\setlength{\parskip}{6pt plus 2pt minus 1pt}
\setlength{\emergencystretch}{3em}  % prevent overfull lines
\setcounter{secnumdepth}{0}
% Pandoc toggle for numbering sections (defaults to be off)
\setcounter{secnumdepth}{0}
% Pandoc header



\begin{document}
\begin{frontmatter}

  \title{rOpenSci tools for textmining open source science literature}
    \author[cstar]{Scott Chamberlain\corref{c1}}
   \ead{scott(at)ropensci.org} 
   \cortext[c1]{Corresponding author}
      \address[cstar]{rOpenSci, Museum of Paleontology, University of California, Berkeley,
CA, USA}
  
  \begin{abstract}
  Corresponding Author:
  
  Scott Chamberlain
  
  rOpenSci, Museum of Paleontology, University of California, Berkeley,
  CA, USA
  
  Email address:
  \href{mailto:scott@ropensci.org}{\nolinkurl{scott@ropensci.org}}
  
  \newpage
  
  Background. xxxx.
  
  Methods. xxxx.
  
  Results. xxxx.
  
  Discussion. xxxx.
  \end{abstract}
  
 \end{frontmatter}


\newpage

\section{Introduction}\label{introduction}

There's likely more than 100 million articles published (source:
Crossref API), representing an enormous amount of knowledge. In addition
to simply reading these articles, they contain a vast trove of
information of interest to researchers.

For example, many researchers are interested in statistical outcomes of
articles: questions about P-values, about effect sizes, and more. With
regard to effect sizes, these are of particular interest, as they are
often combined in meta-analyses to draw broad conclusions about a
particular question.

Text-mining is the broad term associated with pulling information out of
articles. Given the importance of text-mining, good text-mining tools
are needed to make it easier for researchers to do.

Here, we do an overview of text-mining tools in the R programming
language. We do not cover analysis tools per se, but rather those tools
for searching for, acquiring, and ``mashing up'' text.

\section{Digital articles: technical
aspects}\label{digital-articles-technical-aspects}

Of digital articles some of which are available digitally, and some of
which are not. Those that are digital can be split into two groups:
easily machine readable and non-machine readable.

The machine readable articles are those in XML, JSON, or plain text
format. The former two, XML and JSON, are ideal of the machine readable
types because they are structured data, whereas plain text has no
structure - it's simply a long set of characters with line breaks and
spaces in between.

Of the non-machine readable kind, there's PDFs. These can be broken out
into two groups: text based PDFs and scanned PDFs. The former are
converted from digital versions of various kinds (MS Word, OpenOffice,
markdown, etc.), while the latter are PDFs created by scanning in print
articles for which there is no digital version.

\section{Digital articles: the access
landscape}\label{digital-articles-the-access-landscape}

Acces to full-text is the holy grail in text-mining. Some use cases can
get by with article metadata (authors, title, etc.), some with
abstracts, but many use cases need full-text.

The landscape of access to full-text is a extremely hetergeous, with the
majority of variation along the publisher axis. The major hurdle are
paywalls. The majority of articles are published by the big three
publishers - Wiley, Springer, Elsever - and the majority of their
articles are behind paywalls.

A promising sign is that there's an increasing number of open access
publishers. xxxx.

\section{The discovery problem (maybe remove
section)}\label{the-discovery-problem-maybe-remove-section}

xxx

\section{Data sources}\label{data-sources}

There is increasing open source scientific literature content available
online. However, only a small proportion of scientific journals provide
access to their full content; whereas, most publishers provide open
access to their metadata only (most often through Crossref; Table 1).

Table 1. Sources of scientific literature, their content type provided
via web services, whether rOpenSci has an R packages for the service,
and where to find the API documentation.

\begin{longtable}[]{@{}llll@{}}
\toprule
Data Provider & Content Type & rOpenSci Pkg? & API
Documentation\tabularnewline
\midrule
\endhead
Crossref & Metadata only & rcrossref & \footnote{\url{http://api.crossref.org}}\tabularnewline
DataCite & Metadata only & rdatacite & \footnote{\url{https://support.datacite.org/docs/api}}\tabularnewline
Biodiversity Heritage Library & Full content/Metadata & rbhl &
\footnote{\url{http://bit.ly/KYQ1Rd}}\tabularnewline
Public Library of Science (PLoS) & Full text/altmetrics & rplos &
\footnote{\url{http://api.plos.org/solr}}\tabularnewline
Scopus (Elsevier) & Full content/Metadata & fulltext & \footnote{\url{http://bit.ly/J9S616}}\tabularnewline
arXiv & Full content/Metadata & aRxiv & \footnote{\url{https://arxiv.org/help/api/index}}\tabularnewline
Biomed Central (via Springer) & Full content/Metadata & fulltext &
\footnote{\url{https://dev.springer.com/}}\tabularnewline
bioRxiv & Full content/Metadata & fulltext & \footnote{\url{http://www.biorxiv.org/}}\tabularnewline
PMC/Pubmed (via Entrez) & Full content/Metadata & rentrez & \footnote{\url{https://www.ncbi.nlm.nih.gov/books/NBK25500}}\tabularnewline
Microsoft Academic Search & Metadata & fulltext/microdemic & \footnote{\url{https://azure.microsoft.com/en-us/services/cognitive-services}}\tabularnewline
\bottomrule
\end{longtable}

The following is a synopsis of the major data sources and associated R
tools.

\subsection{Crossref}\label{crossref}

Crossref is a non-profit that creates (or ``mints'') Digital Object
Identifiers (DOIs). In addition, they maintain metadata associated with
each DOI. The metadata ranges from simple (including author, title,
dates, DOI, type, publisher) to including number of citations to the
article, as well as references in the article, and even abstracts.

Crossref does have a text-mining opt-in program for publishers. The
result of this is that some publishers deposit URLs for full text
content of their articles. The majority of these links are pay-walled,
while some are open access. Using any of the various tools for working
with Crossref data, you can filter your search to get only articles with
full text links, and further to get only articles with full text links
that are open access.

The main interface for Crossref in R is
\href{https://github.com/ropensci/rcrossref}{rcrossref}. Parallel
interfaces are available in Ruby
(\href{https://github.com/sckott/serrano}{serrano}) and Python
(\href{https://github.com/sckott/habanero}{habanero}).

\subsection{Pubmed}\label{pubmed}

Pubmed is a corpus/website of NIH funded research \ldots{}

\section{How to text mine from R: Three case
studies}\label{how-to-text-mine-from-r-three-case-studies}

\subsubsection{Case study 1}\label{case-study-1}

\subsubsection{Case study 2}\label{case-study-2}

\subsubsection{Case study 3}\label{case-study-3}

\section{Conclusions and future
directions}\label{conclusions-and-future-directions}

xxxx

\section{Acknowledgments}\label{acknowledgments}

xxxx

\section{Data Accessibility}\label{data-accessibility}

All scripts and data used in this paper can be found in the permanent
data archive Zenodo under the digital object identifier (DOI). This DOI
corresponds to a snapshot of the GitHub repository at
\url{https://github.com/ropensci/textmine}. Software can be found at
\url{https://github.com/ropensci/xxx}, xxxx, all under MIT licenses.

\section{References}\label{references}

\end{document}


